\chapter{Einleitung}
Im Rahmen dieser Arbeit erfolgt eine Kombination und Interpretation des $t\bar{t}\gamma$-Produktionswirkungsquerschnitts von einer ATLAS-~\cite{Aaboud:2017era} und CMS-Messung~\cite{Sirunyan:2017iyh} im Rahmen von effektiven Feldtheorien (EFT). Dabei handelt es sich um eine Untersuchung im Hinblick auf Physik jenseits des Standardmodells (engl. \textit{beyond-the-standard-model}, BSM). Effektive Feldtheorien bieten dafür eine gute Grundlage, da sie es ermöglichen eine modellunabhängige Beschreibung neuer Phänomene zu liefern. Zudem können Energieskalen mit einbezogen werden, die mit heutigen Beschleunigern nicht zu erreichen sind. Eine Möglichkeit dies umzusetzen ist mit Hilfe von EFT-Operatoren, deren Kopplungsstärke an Standardmodell-Teilchen mit Wilson-Koeffizienten beschrieben werden kann. Es werden $t\bar{t}\gamma$-Ereignisse untersucht, da verschiedene Faktoren wie die große Top-Masse $m_t$ dafür sprechen, dass es besonders stark an BSM-Physik koppelt. Beide Messungen erfolgten dabei über Lepton+Jets-Endzustände.\\
Ziel dieser Arbeit ist es mit dem EFT\textit{fitter}~\cite{Castro:2016jjv} Einschränkungen für die Wilson-Koeffizienten auf die bei der Produktion von $t\bar{t}\gamma$-Ereignissen beteiligten Operatoren zu finden. Da mit einer steigenden Anzahl an Messungen diese Einschränkungen genauer werden, werden die beiden Messungen von ATLAS und CMS mit dem EFT\textit{fitter} kombiniert.
Um eine Kombination der beiden Messungen überhaupt zu ermöglichen, muss zunächst eine Phasenraumstudie durchgeführt werden. Dies lässt sich darauf zurückführen, dass die gemessenen Phasenräume nicht übereinstimmen und deshalb vereinheitlicht werden müssen. Dies geschieht mit Hilfe von MadGraph5~\cite{Alwall:2014hca}.\\
Im Anschluss an diese Angleichung der Phasenräume kann die Kombination der Messungen erfolgen; dies geschieht mit dem EFT\textit{fitter}. Eine erste Kombination erfolgt unter der Annahme, dass die Unsicherheiten der Messungen nicht korreliert sind. Im Anschluss erfolgt eine Korrelationsstudie, um zu untersuchen, ob eine Korrelation einen Einfluss auf das Ergebnis der Kombination hat. Da CMS bei der Messungen explizit zwischen den Endzuständen mit einem Elektron-Lepton und einem Myon-Lepton unterscheidet, wird sowohl die Korrelation der beiden CMS-Messungen untereinander, als auch zwischen diesen und der ATLAS-Messung untersucht.\\
Zur Bestimmung der Wilson-Koeffizienten wird zu Beginn ein Modell für diese mit MadGraph5 bestimmt und im Anschluss im EFTfitter implementiert. Es dient dazu, den Einfluss der Wilson-Koeffizienten auf das Messergebnis zu implementieren. Durch den Vergleich des Modells mit den Messungen ergeben sich die Grenzen für die beteiligten Wilson-Koeffizienten.
