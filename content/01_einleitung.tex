\chapter{Einleitung}
Ziel dieser Arbeit ist es mit dem EFTfitter\cite{Castro:2016jjv} Einschränkungen für die Wilson-Koeffizienten, der bei der Produktion von $t\bar{t}\gamma$ Ereignissen beteiligten Operatoren zu finden. Da mit einer steigenden Anzahl an Messungen diese Einschränkungen genauer werden, sollen die beiden Messungen des Produktionswirkungsquerschnitts für $t\bar{t}\gamma$ von ATLAS\cite{Aaboud:2017era} und CMS\cite{Sirunyan:2017iyh} mit dem EFTfitter kombiniert werden.\\
Um eine Kombination der beiden Messungen überhaupt zu ermöglichen, muss zunächst eine Phasenraumstudie durchgeführt werden. Dies lässt sich darauf zurückführen, dass die gemessenen Phasenräumen nicht übereinstimmen und deshalb vereinheitlicht werden müssen. Dies geschieht mit Hilfe von MadGraph5\cite{Alwall:2014hca}.\\
Im Anschluss an diese Erweiterung des kleineren Phasenraums kann die Kombination der Messungen erfolgen; dies geschieht mit dem EFT-Fitter. Da nicht ausgeschlossen werden kann, dass die Messungen korreliert sind, wird im Anschluss an eine erste unkorrelierte Kombination eine Korrelationsstudie durchgeführt. Hierbei wird sowohl die Korrelation der beiden CMS-Messungen untereinander, als auch zwischen diesen und der ATLAS-Messung untersucht.\\
Zur Bestimmung der Wilson-Koeffizienten wird zu Beginn ein Modell für diese mit MadGraph berechnet und im Anschluss im EFTfitter implementiert. Durch die Überprüfung des Modells mit den Messungen ergeben sich die Grenzen für die beteiligten Wilson-Koeffizienten.
