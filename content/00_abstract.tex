\thispagestyle{plain}
\section*{Kurzfassung}
In dieser Arbeit wird sich mit einer Kombination zweier Messungen des Produktionswirkungsquerschnitts für $t\bar{t}\gamma$ Ereignisse von ATLAS\cite{Aaboud:2017era} und CMS\cite{Sirunyan:2017iyh} mit dem EFTfitter\cite{Castro:2016jjv} beschäftigt. Anschließend werden mit diesem Einschränkungen für die Wilson-Koeffizienten der beteiligten Operatoren bestimmt.\\
Da die Messungen in unterschiedlichen Phasenräumen erfolgt sind, wird zunächst der Faktor zwischen diesen beiden zu $f=3$ bestimmt. Damit lässt sich die CMS-Messung in den ATLAS Phasenraum erweitern. Die anschließende unkorrelierte Kombination ergibt $\sigma_{\text{Kombi}} = 149,79 \pm 17,57~ \si{\femto\barn}$. Bei einer anschließenden Korrelationsstudie wird festgestellt, dass der kombinierte Wirkungsquerschnitt stark von der Korrelation zwischen der ATLAS und den CMS Messungen abhängt. Daher muss diese zunächst bestimmt werden. Die anschließenden Untersuchung der Wilson-Koeffizienten $C_{tG}$, $C_{tW}$ und $C_tB$ ergibt, dass diese alle mit Null verträglich sind. Die $\SI{68.1}{\percent}$ Intervalle der Koeffizienten $C_{tW}$ und $C_tB$ sind jedoch noch relativ breit und motivieren eine weiter Untersuchung.

\section*{Abstract}
\begin{english}
The abstract is a short summary of the thesis in English, together with the German summary it has to fit on this page.
\end{english}
