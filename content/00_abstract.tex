\thispagestyle{plain}
\section*{Kurzfassung}
Das Ziel dieser Arbeit ist die Kombination zweier Messungen des Produktionswirkungsquerschnitts für $t\bar{t}\gamma$ Ereignisse von ATLAS\cite{Aaboud:2017era} und CMS\cite{Sirunyan:2017iyh} mit Hilfe des framworks EFTfitter\cite{Castro:2016jjv}. Aus den daraus folgenden Ergebnissen werden Einschränkungen für die
Wilson-Koeffizienten der beteiligten Operatoren einer effektiven Feldtheorie bestimmt.\\
Da die Messungen in unterschiedlichen Phasenräumen erfolgt sind, wird zunächst der Faktor zur Vereinheitlichung beider Phasenräume zu $f=3$ bestimmt. Damit lässt sich die CMS-Messung in den ATLAS Phasenraum erweitern. Die anschließende unkorrelierte Kombination ergibt einen Wirkungsquerschnitt von $\sigma_{\text{Kombi}} = 149,79 \pm 17,57~ \si{\femto\barn}$. Bei einer anschließenden Korrelationsstudie wird festgestellt, dass der kombinierte Wirkungsquerschnitt stark von der Korrelation zwischen der ATLAS und den CMS Messungen abhängt. Daher muss diese zunächst bestimmt werden. Die anschließenden Untersuchung der Wilson-Koeffizienten $C_{tG}$, $C_{tW}$ und $C_{tB}$ ergibt, dass diese alle mit Null verträglich sind. Die $\SI{68.1}{\percent}$ Intervalle der Koeffizienten $C_{tW}$ und $C_{tB}$ sind jedoch relativ breit und motivieren eine weitere Untersuchung.
\section*{Abstract}
\begin{english}
  The aim of this work is to combine two production cross section measurements for $t\bar{t}\gamma$ events of ATLAS\cite{Aaboud:2017era} and CMS\cite{Sirunyan:2017iyh} using the framework EFTfitter\cite{Castro:2016jjv}. The results yield restrictions for the
  Wilson coefficients of the operators involved in an effective field theory.\\
  Since the measurements were carried out in different phase spaces, firstly the factor for standardizing both phase spaces is determined to be $f=3$. This allows to transform the CMS measurement into the ATLAS phase space. The uncorrelated combination gives a cross section of $\sigma_{\text{Kombi}} = 149,79 \pm 17,57~ \si{\femto\barn}$. In a subsequent correlation study it is found that the combined cross section is strongly influenced by the correlation between the ATLAS and CMS measurements. Therefore, this must first be determined. Based on this the investigation of the Wilson coefficients $C_{tG}$, $C_{tW}$ and $C_{tB}$ shows that they are all compatible with zero. The $\SI{68.1}{\percent}$  intervals of the coefficients $C_{tW}$ and $C_{tB}$ are, however, relatively broad and motivate further investigation.
\end{english}
