\thispagestyle{plain}
\section*{Kurzfassung}
Das Ziel dieser Arbeit ist die Kombination zweier Messungen des Produktionswirkungsquerschnitts für $t\bar{t}\gamma$-Ereignisse von ATLAS~\cite{Aaboud:2017era} und CMS~\cite{Sirunyan:2017iyh} mit Hilfe des frameworks EFTfitter\cite{Castro:2016jjv}. Dafür werden Ereignisse betrachtet, bei denen in Gluonfusion ein $t\bar{t}$-Paar entsteht und an einem Top-Quark ein Photon abgestrahlt wird.
Da die Messungen in unterschiedlichen Phasenräumen erfolgt sind, wird zunächst der Faktor zur Vereinheitlichung beider Phasenräume zu $f\approx 3$ bestimmt. Damit lässt sich die CMS-Messung in den ATLAS-Phasenraum extrapolieren. Die anschließende unkorrelierte Kombination ergibt einen Wirkungsquerschnitt von $\sigma_{\text{Kombi}} = 150 \pm 18~ \si{\femto\barn}$. Bei einer anschließenden Korrelationsstudie wird festgestellt, dass der kombinierte Wirkungsquerschnitt stark von der angenommenen Korrelation zwischen der ATLAS- und der CMS-Messung abhängt. Daher muss die Korrelation zunächst bestimmt werden. Aus den daraus folgenden Ergebnissen werden Einschränkungen für die
Wilson-Koeffizienten der beteiligten Operatoren einer effektiven Feldtheorie bestimmt. Die anschließende Untersuchung der Wilson-Koeffizienten $C_{tG}$, $C_{tW}$ und $C_{tB}$ ergibt, dass diese alle innerhalb des $1\sigma$-Intervalls mit Null verträglich sind.
