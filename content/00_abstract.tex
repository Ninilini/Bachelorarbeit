\thispagestyle{plain}
\section*{Kurzfassung}
Das Ziel dieser Arbeit ist die Kombination zweier Messungen des Produktionswirkungsquerschnitts für $t\bar{t}\gamma$ Ereignisse von ATLAS\cite{Aaboud:2017era} und CMS\cite{Sirunyan:2017iyh} mit Hilfe des framworks EFTfitter\cite{Castro:2016jjv}. Aus den daraus folgenden Ergebnissen werden Einschränkungen für die
Wilson-Koeffizienten der beteiligten Operatoren einer effektiven Feldtheorie bestimmt.\\
Da die Messungen in unterschiedlichen Phasenräumen erfolgt sind, wird zunächst der Faktor zur Vereinheitlichung beider Phasenräume zu $f=3$ bestimmt. Damit lässt sich die CMS-Messung in den ATLAS Phasenraum erweitern. Die anschließende unkorrelierte Kombination ergibt einen Wirkungsquerschnitt von $\sigma_{\text{Kombi}} = 149,79 \pm 17,57~ \si{\femto\barn}$. Bei einer anschließenden Korrelationsstudie wird festgestellt, dass der kombinierte Wirkungsquerschnitt stark von der Korrelation zwischen der ATLAS und den CMS Messungen abhängt. Daher muss diese zunächst bestimmt werden. Die anschließenden Untersuchung der Wilson-Koeffizienten $C_{tG}$, $C_{tW}$ und $C_{tB}$ ergibt, dass diese alle mit Null verträglich sind. Die $\SI{68.1}{\percent}$ Intervalle der Koeffizienten $C_{tW}$ und $C_{tB}$ sind jedoch relativ breit und motivieren eine weitere Untersuchung.
